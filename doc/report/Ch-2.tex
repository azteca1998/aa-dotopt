\chapter{Project configuration}


\section{Python's C API}

    Python's runtime is written in C, and it exposes an API to interact with it.
Using that api, it's perfectly possible to use Python without writing a single
line of Python.

    In our case, the Python C API allows us to write fast code in C and
compatible languages, while still having the ease of use that Python provides.

    The same approach is used by NumPy, the numerical Python library. It is
implemented in C while being usable from within Python with impressive results.

    There are two main ways to interact between C and Python: embedding and
making a module. In this project we use both, but embedding is only used when
benchmarking.

    Making a module is as simple as declaring an exported function with a
specific name, which creates the module that will be used by Python:

\begin{lstlisting}[language=C]
    static struct PyModuleDef module = {
        PyModuleDef_HEAD_INIT,
        .m_name = "dotopt",
        .m_doc = "",
        .m_size = -1,
        .m_methods = methods,
    };

    PyMODINIT_FUNC PyInit_dotopt()
    {
        return PyModule_Create(&module);
    }
\end{lstlisting}

    To use NumPy from C, it must also be initialized from within the module, so
the initialization function 


\section{Benchmarking}


\section{Testing}
